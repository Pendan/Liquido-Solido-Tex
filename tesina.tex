\documentclass[a4paper,11pt, notitlepage]{report}

\usepackage[italian]{babel}
\usepackage[T1]{fontenc}
\usepackage[utf8]{inputenc}
\usepackage{graphicx}
\title{The Blume-Emery-Griffiths model and multicritical points}
\author{Federica Minozzi - Statistical Mechanics}}
\date{}

\begin{document}
\maketitle

Phase transitions can be classified into:
\begin{itemize}
\item first order phase transitions: discontinuities in the first order derivative of the free energy occur;
\item second order (or continuous) phase transitions: discontinuities are found in the second order derivative. Examples of this kind of transition are superconducting, superfluid and ferromagnetic transitions. 
\end{itemize}
For phase transitions which take place as a consequence of changes in a single variable, the temperature T, in the phase diagram there will be a critical point, which is a special combination of pressure and temperature which separates the low temperature ordered (dense) from the high temperature disordered (gasous) phases. In a neighbourhood of the critical point, the two phases are almost indistinguishable. \\
Moreover, one can define a multicritical point, which characterizes a continuous phase transition. In order for a critical point to appear, the thermodynamic parameters involved in the transition must be at least two. \\
In particular, we will have a tricritical point where a three-phase coexistence line terminates. It will connect a $\lambda$ - line, which is  a second order transition line, to a a first order line. \\

An example of a thermodynamic system exhibiting a tricritical point is that of mixtures of $He^3$ and $He^4$. In this case, the $\lambda$ - line will be associated to the transition of $He^4$ from a normal fluid to a superfluid phase. The temperature of this transition is depressed by the mixture with $He^3$. \\

\begin{figure}
\begin{center}
\includegraphics[width=10cm]{p0201-sel.eps}
\end{center}
\caption{Temperature-composition phase diagram for the misture of $He^3$ and $He^4$.}
\label{fig:dessin}
\end{figure}

The tricritical point will connect this second order transition with the phase separation into $He^3$ - rich and $He^3$ - poor phases. 
The phase diagram of this system is successfully described by the Blume-Emery-Griffith spin-lattice model. Let's assume that at each site $I$ on a lattice there is a spin variable $S_I$ whose possible values will be $-1,0,+1$. The order parameter for the model will be the uniform average spin $<SI>$. \\
Note that this kind of order parameter is not able to describe the complex superfluid order parameter. In fact, order parameters usually arise from symmetry breaking, but they can also be defined for non-symmetry-breaking transitions such as superfluid and ferromagnetic. In these cases, the order parameter may be a complex number or a vector. However, the distinction between the two types or order is not relevant to our model.\\
We can associate $<SI_2>$ to the density of $He^4$ atoms in the mixture and consequently $1 - <SI_2>$ will be the density of $He^3$ atoms. Thus, the Blume-Emery-Griffith Hamiltonian will be written as:
\begin{equation}
H = - J \sum_{II'} S_IS_{I'} - K \sum_{II'} S_I^2SI'^2 + \triangle \sum_I SI^2
\end{equation}
Actually, the first order line near the tricritical point corresponds to the cohexistence of three phases: a paramagnetic and two degenere antiferromagnetic phases. The degeneracy can be solved by applying a field $h$ conjugate to the order parameter $S_I$, so that the Hamiltonian becomes:
\begin{equation}
H = - J \sum_{II'} S_IS_{I'} - K\sum_{II'} S_I^2S_I'^2 + \triangle \sum_I S_I^2 - h \sum_I S_I 
\end{equation}
The phase diagram for the Blume-Emery-Griffith model in the $T - \triangle - h$ plane, shown in figure, allows to show the tricritical point in a three dimensional space as the meeting of three second order lines. \\  

\begin{figure}
\begin{center}
\includegraphics[width=10cm]{p0203-sel.eps}
\end{center}
\caption{Phase diagram for the BEG model in the $T - \triangle - h$ plane. The three $\lambda$-line terminating planes of phase coexistence meet at the tricritcal point TP. Three phases coexist along the double line D which is a first-order line.}
\label{fig:dessin}
\end{figure}

Other kinds of multicritical points are represented by the bicritical and tetracritical points. Those are generated by breaks of symmetry as a consequence of anistropies that may favour spin alignment along particular lattice directions: examples are present in many antiferromagnets in which the lattice anisotropy is not strong enough to enforce complete spin allignment. If $\phi_1$ and $\phi_2$ are the Ising order parameters of the system, the Landau free energy will be written as:
\begin{equation}
f = \frac{1}{2}r(\phi_1^2 + \phi_2^2) - \frac{1}{2}g(\phi_1^2 - \phi_2^2) + u_1\phi_1^4 + u_2\phi_2^4 + 2u_12\phi_1^2\phi_2^2
\end{equation}
We can have different cases:
\begin{itemize}
\item $g=0$, $u1=u2=u12$: the model reduces to a system with isotropic interactions and a vector order parameter $\phi=(\phi_1,\phi_2)$;
\item $g>0$: $\phi_1$ will oder before $\phi_2$ and there will be an ordered phase with $\phi_1 \neq 0$ and $\phi_2 = 0$, which we will call antiferromagnetic phase;
\item $g<0$: $\phi_2$ will order before $\phi_1$ and there will be an ordered phase with $\phi_1 = 0$ and $\phi_2 \neq 0$, which we will call spin-flop phase, where the definition of spin-flop transition is soon to be explained.
\end{itemize}
Let's now distinguish two cases depending on the relative magnitudes of the fourth order potentials.

\begin{itemize}
\item When $u_1u_2 < u_{12}^2$ there will be a first-order line along $g=0$, $r<0$ separating the antiferromagnetic and spin-flop phases. At the point $r=0$, $g=0$, two second-order lines meet. This will be thus called a bicritical point. \\
A bicritical point represent the termination of a first-order transition, which may occur in antiferromagnets. In fact, the zero-field state of the material is antiferromagnetic with the staggered magnetization $m_s$ aligned along the anisotropy axis. When a field parallel to the anisotropy axis is applied, it will initially increase the magnetization in one sublattice over the other, without changing the direction of $m_s$. \\
At a critical value of the field, however, the direction of $m_s$ spontaneously flips from parallel to perpendicular to the field: this is called spin-flop transition and it terminates at a bricritical point. \\
For $H<H_{BP}$ (before of the magnetization flip) only the parallel components order, while for $H>H_{BP}$ (after the magnetization flip) only the perpendicular components order, but for $(H, T) = (H_{BP}, T_{BP})$ both the parallel and perpendicular components of $m_s$ are critical. 
\item When $u_1u_2 > u_{12}^2$ the antiferromagnetic and spin-flop phases will be separated by a second-order line, but in between there will be an intermediate phase with both $\phi_1$ and $\phi_2$ different from zero. Thus, the point $r=0$, $g=0$ will be the meeting point of four second-order lines, and it will be consequently called a tetracritical point. 
\end{itemize}
The phase boundaries near either the britical or the tetracritical point predicted by mean-field theory will be straight lines. The figure below shows the phase diagram in both the cases of a bicritical and a tetracritical point. 

\begin{figure}
\begin{center}
\includegraphics[width=10cm]{p0204-sel.eps}
\end{center}
\caption{Phase diagram showing $(a)$ a bicritcal point (BP) when $u_1u_2 < u_{12}^2$ and $(b)$ a tetracritical point (TP) when $u_1u_2 > u_{12}^2$.}
\label{fig:dessin}
\end{figure}

\end{document}


